\begin{abstract}
El objetivo de este documento es describir el problema presentado, y las diversas soluciones que se fueron realizando utilizando distintas tecnicas de programaci\'on, para la solucion exacta, o diversas heur\'isticas para resolver el problema de forma polinomial, dado que no existe solucion exacta con complejidad polinomial.\\

El problema que se nos presenta es, dado un grafo $G=(V, E)$, dos v\'ertices $u,v \in V$, dos funciones $w_1$ y $w_2$ de costo asociadas a las aristas del grafo, y un valor $K \in \mathbb{R}$, se pide resolver el problema de Camino Acotado de Costo Minimo (CACM), el cual consiste en encontrar un camino P entre $u, v$ tal que el costo del camino respecto a la funcion $w_2$ sea minimo y valga la condicion del costo del camino $w_1 \leq K$.\\

Se nos requiri\'o modelar situaciones de la vida real con CACM y luego implementar diversas soluciones para este problema, preferimos C y C++ como lenguajes de programaci\'on de este trabajo practico y en el fueron implementados los siguientes algoritmos:
\begin{itemize}
	\item Soluci\'on exacta (C/C++)
	\item Heur\'istica constructiva golosa (C++)
	\item Heur\'istica de busqueda local (C++)
	\item Metaheur\'istica GRASP (C++)
\end{itemize}

Todas las soluciones listadas arriba fueron testeadas con diversos casos de testing y ser\'an claramente documentadas en las secciones de este documento.
Para cada punto se detalla el algoritmo, se establece una cota teorica sobre la complejidad temporal, y se brindan ejemplos de funcionamiento.
En el caso de la soluci\'on exacta ademas se realizaron pruebas de performance y son presentadas con graficos que acompan\~an dichos experimentos.
Para las heur\'isticas golosas y de busqueda local se realizo un an\'alisis de las instancias para las cuales la soluci\'on obtenida es optima y para cuales no se alcanza dicho optimo, asimismo donde fue posible, se indico que tan mala puede ser la solucion obtenida respecto de la solucion optima.
Finalmente se realiz\'o una experimentacion general indicando en todos los algoritmos la performance temporal obtenida segun el taman\~o de entrada y donde corresponda los parametros(GRASP por ejemplo), ademas fueron comparadas las soluciones obtenidas por el algoritmo exacto y cada una de las heur\'isticas desarrolladas. 
\end{abstract}

