\section{Conclusion}
En este trabajo, al desarrollar varios algoritmos que solucionan, de forma exacta o aproximada un problema dado, pudimos observar la performance, tanto temporal como de optimalidad de varias tecnicas para el desarrollo de heuristicas, analizar varios aspectos de las mismas, como por ejemplo que familias de grafos son buenas o malas para cada una, deducir, variando la densidad de las instancias de prueba, para que familia convenia aplicar cual u otra heuristica, analizar las variaciones absolutas y relativas de la funci\'on target a medida que avanzan las iteraciones de ciertas heuristicas, en busqueda local por ejemplo: probar con distintas combinaciones de vecindad y solucion inicial.\\
\vspace{0.5cm}
Finalmente, en la secci\'on de experimentacion general se explican la generacion aleatoria de grafos, los scripts utilizados para realizar los experimentos, el modelo de puntajes porcentuales y estimadores que utilizamos para realizar las comparaciones de optimalidad, los graficos realizados y sus referencias, y para 3 tipos distintos de densidades de grafos generados aleatoriamente, se realizaron 2 experimentos y se escribieron conclusiones acerca de sus resultados. Estos dos experimentos a grandes rasgos eval\'uan, el primero, el porcentaje de veces que las heuristicas dan la solucion optima y que tan lejos de ella estan junto con la comparacion del tiempo de ejecucion entre el algoritmo de la heuristica y el algoritmo de solucion exacta, y el segundo, entre las 3 heuristicas, para los valores m\'aximos de nodos y densidad que pudimos evaluar con el hardware con el que contamos, con que porcentaje cada heuristica da el m\'inimo valor de la solucion(por ende el mas cercano al \'optimo), la que tenga el porcentaje mas alto, ser\'a la que mas se acerque a una solucion exacta.
\vspace{0.5cm}
Si tuvieramos que elegir una sola heuristica para solucionar el problema de forma aproximada, elegiriamos la heuristica golosa, que demostr\'o tener la mayor performance y tiempos de ejecuci\'on aceptables(salvo algunos casos, ver seccion de exp. general de golosa, grafos de baja densidad.). La heuristica de b\'usqueda local alimentada con dijkstra sobre $w_1$ provee soluciones lejanas al optimo, la utilizariamos solo para refinar soluciones factibles de buena calidad ya obtenidas, no como heuristica para solucionar el problema. La metaheuristica GRASP, dado lo que dijimos antes, deberia ser la mejor, dado que la soluci\'on golosa es muy buena y la busqueda local podr\'ia refinarla, pero no es as\'i, vimos que variando el parametro beta de la metaheuristica tanto en RCL por valor o cantidad, al disminuir este parametro la golosa randomizada se acerca a la golosa pura y como vimos en la seccion de busqueda local(greedy como solucion inicial), en promedio nunca hay ninguna mejora a nivel local para las soluciones golosas.Al aumentar el parametro, la solucion inicial baja el porcentaje de aciertos con el optimo, asi que mantuvimos el parametro a niveles bajos, y no obtuvimos mejora significativa, en el mejor de los casos, GRASP da el mismo porcentaje de aciertos respecto al optimo que la heuristica golosa, pero luego de ajustar mucho los parametros y realizar muchisimos experimentos, pudimos igualar el algoritmo goloso en cantidad de aciertos en la solucion optima pero reduciendo la lejania entre la solucion optima y la solucion provista por GRASP.