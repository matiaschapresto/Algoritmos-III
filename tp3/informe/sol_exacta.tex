\section{Algoritmo exacto para la resoluci\'on de CACM}

\subsection{Notaci\'on} \label{exacto:notacion}

\begin{itemize}
	\item $G$ $=$ Grafo al que se le quiere encontrar el \emph{CACM}
	\item $P$ $=$ El camino a encontrar
	\item $v, w$ $=$ Los nodos de origen y destino del camino $P$ a encontrar
	\item $K$ $=$ Cota superior del peso $\omega_1$ del camino $P$ a encontrar
\end{itemize}

\subsection{Explicaci\'on detallada del algoritmo propuesto} \label{exacto:explicacion}

\subsubsection{Descripci\'on}

El algoritmo que se propone para encontrar la soluci\'on exacta al poblema del \emph{CACM}, consiste en encontrar todos los caminos simples de $v$ a $w$
y encontrar el camino $P$ que tenga el menor peso $\omega_2$ pero que $\omega_1$ no supere a $K$.

Se va a realizar un backtracking que s\'olo chequee los caminos que no superen a $K$ el peso de $\omega_1$, y que descarte los caminos en cuanto
detecta que no puede mejorar una soluci\'on que ya tenga almacenada como candidato a ser la de menor peso $\omega_2$.

El algoritmo ser\'a una funci\'on recursiva, pas\'andole como argumentos el grafo, los nodos de origen ($u$ el cual en la primer llamada va a ser $v$,
el nodo origen del problema) y destino ($w$), el limite K, y un camino encontrado hasta el momento, que va desde el nodo de origen del problema ($v$), al nodo de orgen pasado a la funci\'on ($u$).

En cada llamada recursiva, la funci\'on recorre todos los adyacentes a $u$, y por cada uno (nombremoslo $z$) llama recursivamente quitando el v\'ertice $(u,z)$ al grafo,
,indic\'andole como nodo de origen a $z$, y agregando al camino encontrado hasta el momento el nodo $z$. Luego se queda con el mejor camino que encontr\'o entre cada uno de los adyacentes.

\subsubsection{Algoritmo}

\begin{algorithm}[!h]
\caption{cacm\_exacto} \label{exacto:pseudo}
\end{algorithm}
\begin{algorithmic}[1]
	\Require \emph{grafo}: el grafo al cual se tiene que encontrar el \emph{CACM}
	\Require \emph{u}: el nodo de origen
	\Require \emph{w}: el nodo de destino
	\Require \emph{K}: cota del peso $\omega_1$
	\Require \emph{solucion}: posible soluci\'on que se tiene hasta el momento. Por aca se recibe un camino posible y se actualiza en cada llamada recursiva con el mejor camino encontrado. Cuando terminen todas las llamadas recursivas, se retorna por aca el \emph{CACM} encontrado. En la primera llamada se debe llamar con una soluci\'on que s\'olo contenga al nodo de partida $v$.
	\Statex
	\Ensure Retorna $verdarero$ si encontr\'o un camino de $u$ a $w$, $falso$ si no encontr\'o camino.
	\Statex
	\Procedure{cacm\_exacto}{$Grafo$: grafo, $Nodo$: u, $Nodo$: w, $Real$: K, $Camino$: solucion}{$\to$ $bool$}
		\If{u $=$ w}
			\State return $verdadero$
		\EndIf
		
		\State solucion\_nueva $\gets$ $[]$
		\State solucion\_mejor $\gets$ $[]$

		\State \Call{MarcarVisitado}{grafo, u, true}
		\For{z $\gets$ \Call{adyacentes}{grafo, u}}
		\If{$\neg$ \Call{EstaVisitado}{grafo, z}}
			\State peso1 $\gets$ $\omega_1({u, z})$
			\State peso2 $\gets$ $\omega_2({u, z})$
			\If{$peso1 < K$ $\land$ (mejor\_solcion $=$ $[]$ $\lor$ $peso2 < \omega_2(solucion\_mejor))$}
				\State \Call{quitar\_arista}{grafo, u, z}
				\State solucion\_nueva $gets$ $[z]$
				\State encontre $\gets$ \Call{cacm\_exacto}{grafo, z, w, K $-$ $peso1$, solucion\_nueva}
				\If{encontre $\land$ (mejor\_solucion $=$ $[]$ $\lor$ $\omega_2(solucion\_nueva) < \omega_2(solucion\_mejor)$)}
					\State mejor\_solucion $\gets$ solucion\_nueva
				\EndIf
				\State \Call{agregar\_arista}{grafo, u, z, peso1, peso2}
			\EndIf
		\EndIf
		\EndFor
		\State \Call{MarcarVisitado}{grafo, u, false}
		\If{solucion\_mejor $\neq$ $[]$}
			\State solucion $\gets$ solucion $+$ solucion\_mejor
		\EndIf
		\State return solucion\_mejor $\neq$ $[]$
	\EndProcedure
\end{algorithmic}

\subsection{Complejidad temporal para el peor caso} \label{exacto:complejidad}

Como el Algoritmo (\ref{exacto:pseudo}) se puede ver que depende de la cantidad de aristas de cada nodo (ya que para cada nodo, recorre todos los adyacentes), dados dos grafos
con igual cantidad de nodos, el algoritmo realizar\'a m\'as iteraciones con el grafo que contenga m\'as aristas. Por ende, los peores casos son los grafos completos ($K_n$),
que son los grafos con mayor cantidad de aristas.

\subsection{Experimentaci\'on de p\'erformance} \label{exacto:experimentacion}

