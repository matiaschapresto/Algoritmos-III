\section{Reentrega: Informe de modificaciones}
En esta secci\'on se detallaran brevemente, esperamos no olvidarnos de ninguna, las modificaciones realizadas con respecto a la primera entrega, decidimos utilizar un sistema de tickets que nos permitio \texttt{trackear} cada modificacion que realizamos. A continuacion se listaran las modificaciones:
\begin{itemize}
	\item \textbf{Exacto: } \texttt{Complejidad: Explicar porque las marcas hacen que el algoritmo funcione}
	\item \textbf{Exacto: } \texttt{Cambiar Kn por acotar debidamente para grafos G genericos}
	\item \textbf{Exacto: } \texttt{Falta caso base T(1) en analisis complejidad}
	\item \textbf{Exacto: } \texttt{Fix calculo de complejidad, ver pagina 9 de la devolucion.}
	\item \textbf{Exacto: } \texttt{Poda 1: Validacion de existencia de camino factible}
	\item \textbf{Exacto: } \texttt{Poda 2: Finalizar la busqueda si se va a realizar una llamada sobre una rama no factible}
	\item \textbf{Exacto: } \texttt{Poda 3: Verificacion de optimalidad de la solucion en construccion}
	\item \textbf{Golosa: } \texttt{Descripcion, Correctitud(siempre hay solucion si hay factible), PseudoCodigo y Complejidad Teorica}
	\item \textbf{Golosa: } \texttt{Familias de Grafos Malas para esta heuristica}
	\item \textbf{Golosa: } \texttt{Experimentaciones y graficos de performance}
	\item \textbf{Golosa: } \texttt{Experimentacion de optimalidad}
	\item \textbf{Bqlocal: } \texttt{Comparacion de mejora evolutiva de iteraciones segun diferentes soluciones iniciales}
	\item \textbf{Bqlocal: } \texttt{Experimentacion rendimiento}
	\item \textbf{Bqlocal: } \texttt{Medicion evolutiva de mejora en iteraciones de una instancia}
	\item \textbf{Bqlocal: } \texttt{Preprocesar listas de vecinos en comun}
	\item \textbf{Bqlocal: } \texttt{Combinar los 3 criterios de bqlocal en una vecindad unificada}
	\item \textbf{Bqlocal: } \texttt{Reescribir complejidad teorica}
	\item \textbf{Bqlocal: } \texttt{Pseudocodigo mas acorde a la combinacion de vecindades}
	\item \textbf{Bqlocal: } \texttt{Definicion bien las vecindades por separado}
	\item \textbf{Bqlocal: } \texttt{Explicar con palabras los 3 modos de vecindad especificados con simbolos}
	\item \textbf{Bqlocal: } \texttt{Nivel de optimalidad de las soluciones}
	\item \textbf{Bqlocal: } \texttt{Ejemplos de familias de grafos malas para esta heuristica}
	\item \textbf{Bqlocal: } \texttt{Comparacion de rendimiento y optimalidad sobre distintas vecindades}
	\item \textbf{Bqlocal: } \texttt{Analisis de cantidad de iteraciones sobre greedy como sol inicial}
	\item \textbf{Grasp: } \texttt{Descripcion, planteo de la RCL, analisis del parametro}
	\item \textbf{GRASP: } \texttt{Complejidad y graficos de complejidad}
	\item \textbf{GRASP: } \texttt{Experimentacion de optimalidad}
	\item \textbf{Generador de grafos: }\texttt{ Explicacion de la generador aleatoria de grafos}
	\item \textbf{Experimentacion: } \texttt{Explicar scripts de performance y optimalidad}
	\item \textbf{Experimentacion: } \texttt{Realizar la experimentacion de comparacion entre todos los algoritmos}
	\item \textbf{Experimentacion: } \texttt{Realizar experimentacion entre las heuristicas unicamente para instancias mas grandes}
\end{itemize}

A rasgos mas generales, las secciones de heuristica golosa, heuristica de busqueda local, metaheuristica GRASP y experimentacion general fueron modificadas casi por completo o reescritas en su totalidad. La seccion de la vida real no tuvo cambios, la seccion del algoritmo exacto tuvo cambios menores que incluyen informacion acerca de las podas realizadas y correcciones pedidas en la devolucion.