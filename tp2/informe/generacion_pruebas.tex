\section{Ap\'endice: Generaci\'on de casos de prueba} \label{casos_de_prueba}

Para el testeo de los algoritmos y la medici\'on de tiempos en funci\'on de la entrada, se program\'o una utilidad para generar los casos de prueba.

Los n\'umeros aleatorios que se generaron en los casos, se hizo con la funci\'on \emph{random()} de C ( \href{http://linux.die.net/man/3/random}{http://linux.die.net/man/3/random} ) usando como semilla el tiempo en microsegundos.

Para el ejercicio 2, se le indica al programa la cantidad de ciudades, la cual se fue incrementando para la medici\'on del tiempo, y la cantidad de centrales se indicaba como un n\'umero aleatorio entre 1 y la cantidad de ciudades.

As\'i se crearon muchos casos donde se fue incrementando el tama\~no de la entrada y ejecutando varios tests de un mismo tama\~no y varias veces el mismo test para obtener un promedio del tiempo que tarda en resolvero y descartar algunas imperfecciones que puedan surgir por el entorno donde se est\'a midiendo los tiempos, como puede ser que justo se ejecute otra tarea.

Para los ejercicios 1 y 3, se programaron dentro de las clases Main de los diferentes ejercicio diferentes metodos que se encargan de la codificacion y decodificacion de entrada y salida tanto como de la generacion de tests aleatorios con distribucion uniforme. Se indica por parametro al ejecutable que se desea realizar.